
\chapter{Package implementation in R}
\label{cha:R}

Once understood what is a Bitcoin Cash, how works the technology behind it and the
reasons why it was created. Its time to start and read some data from the blockchain
in order to put into practice what we have learned in the above chapters. Moreover,
after the data is imported, its a good practice to analyze the data in such a way as to 
get usefull information out of it. To do so, the R programming language will be used.

\section{R}
\label{sec:enivirionment}

R is an Open Source software for statistical computing(linear and nonlinear
 modelling, classical statistical tests, time-series analysis, classification,
 clustering) and graphics. One of R’s strengths is the ease with which well-designed 
 publication-quality plots can be produced, including mathematical symbols and 
 formulae where needed.\\
R is designed around a true computer language, and it allows users to add additional 
functionality by defining new functions. Much of the system is itself written in the 
R which makes it easy for users to follow the algorithmic choices made. But, for
computationally-intensive tasks, C, C++ and Fortran code can be linked and called 
at run time. Furthermore, thanks to its Open Source nature, it can be easily extended
via "Packages" even written by the user himself.\cite{r}

\section{Feasibility study}
\label{sec:study}

The goal for this project is to extend the R lenguage with a package which allow us
to read and analyze the information into the blockchain Bitcoin Cash.\\
To make it in the best way, firstly was done a research of the existing 
implementations, unfortunately, whitout any result. But, it brings us two 
completely differnt ways to realize the package. \\
The first one was to turn the used device into a node of the Bitcoin Cash network.
This lead user to download all the blockchain and make all the queries locally.
As a result, the queries will be very fast and all the data of the blockchain can
be manipulated as you desire. But, on the other hand, storing all the blockchain 
requires a lot of space on you device (around 143 GigaByte on 13/9/2019). So making
a package of that size doesn't seem to be the best option.\\
For theese reasons, the second way was choosen. Altough its slowest implementation,
it avoids the need to become a node of the network, making the installation of the
package very simple and fast. That was obtained by relying on a service provider.

\section{Package integration}
\label{sec:integration}



\subsection{Commands}
\label{sec:commands}

\section{Usage samples}
\label{sec:sample}
