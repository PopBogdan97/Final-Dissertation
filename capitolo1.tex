\chapter{Blockchain}
\label{cha:blockchain}

The application of the blockchain in multiple fields is gaining more and more 
traction today, until to be considered by some as the new Internet.
But actually, the main concept behind this revolutionary idea was described as 
early as 1991 in an accademic paper\footnote{\url{https://www.anf.es/pdf/Haber_Stornetta.pdf}}
published by Stuart Haver and W. Scott Stornetta. With their solution, the two 
scientists aim to Time-Stamp documents, in order to certify when a document
was created of last changed. The system used a cryptographically secured 
chain of blocks to store the time-stamped documents and in 1992 Merkle 
trees were incorporated to the design, making it more efficient by 
allowing several documents to be collected into one block.\cite{binancevision}
However, this is not enought. Indeed, to have a fully working blockchain 
we need to add more components.




\section{Distributed ledger}
\label{sec:ledger}

As said in a report from the UK Government, "A distributed ledger is 
essentially an asset database that can be shared across a network of 
multiple sites, geographies or institutions. All participants within a 
network can have their own identical copy of the ledger"\cite{ukgov}.
Due to this technology, there aren't either central administrators or 
centralized data storages. To have this mechanism working, we need both
a Peer-to-Peer Network\footnote{the peers are computer systems connected
to each other via the Internet, the data can be shared directly between 
systems on the network.} and a Consensus Algorithm \ref{sec:consensus}. 

\section{Consensus Algorithms}
\label{sec:consensus}

To clarify the benefits of this type of algorithms we must start with a 
dilemma as known as "The Byzantine Generals Problem", in whom, each 
general has its own army and that each group is situated in different 
locations around the city they intend to attack. The generals need to 
agree on either attacking or retreating. It does not matter whether 
they attack or retreat, as long as all generals reach consensus, 
i.e., agree on a common decision in order to execute it in coordination. 
Therefore, we may consider the following requirements:
\begin{itemize}
    \item Each general has to decide: attack or retreat (yes or no)
    \item After the decision is made, it cannot be changed
    \item All generals have to agree on the same decision and execute it in a synchronized manner
\end{itemize}
The aforementioned communication problems are related to the fact that 
one general is only able to communicate with another through messages, 
which are forwarded by a courier. Consequently, the central challenge 
of the Byzantine Generals’ Problem is that the messages can get somehow 
delayed, destroyed or lost.

In addition, even if a message is successfully delivered, one or more 
generals may choose (for whatever reason) to act maliciously and send 
a fraudulent message to confuse the other generals, leading to a total 
failure.

If we apply the dilemma to the context of blockchains, each general 
represents a network node, and the nodes need to reach consensus on 
the current state of the system. Putting in another way, the majority 
of participants within a distributed network have to agree and execute 
the same action in order to avoid complete failure.\cite{byzantine}\cite{binancevision}
There are various implemetation of the mechanism through which a blockchain 
network reach consensus.


\subsection{Proof of Work}
\label{sec:pow}

Due to Bitcoin \ref{cha:bitcoin}, the most known is the Proof of 
Work (PoW) algorithm, in whom is utilized for block generation. 
The process of generating correct proofs in order to add a block 
to the blockchain is known as “mining” and the individuals that 
participate in the mining process are known as “miners”.\cite{consensusmedium}

The PoW mining involves numerous hashing attempts, so more computational 
power means more trials per second. In other words, miners with a high 
hash rate have better chances to find a valid solution for the next 
block. The PoW consensus algorithm makes sure that miners are only able 
to validate a new block and add it to the blockchain if 
the distributed nodes of the network reach consensus and agree that the 
block hash provided by the miner is a valid proof of work.\cite{binancevision}

\subsection{Proof of Stake}
\label{sec:pos}

The Proof of Stake (PoS) consensus algorthm 
is relatively different and a new way to generate and append blocks in a 
blockchain. It was developed in 2011 as alternative to PoW, which requirements
a massive amout of energy to make it work. In fact, according to the 
bitcoin energy consumption tracker at Digiconomist\footnote{\url{https://digiconomist.net/bitcoin-energy-consumption}}, 
bitcoin currently consumes 66.7 terawatt-hours per year. That’s comparable 
to the total energy consumption of the Czech Republic, a country 
of 10.6 million people.

Basically, the mining process is replaced with a mechanism where blocks are
validated according to the stake of the participants, also known as "validadors".
The criteria used to choose the validator depends on the proof of stake system
but mainly, the choiche is based on the economic stake in the network of 
each node.\cite{binancevision}\cite{consensusmedium}

\subsection{Proof of Burn}
\label{sec:pob}

Also the Proof of Burn (PoB) algortigm is an approach to avoid the massive 
waste of energy used for hashing. Indeed , with this technique, the mining
process is replaced with a greener one. Where, the "miners" of the PoB 
coins will send coins to an  uspendable address, known as "eater address".
Even this transactions are recorded on the blockchain ensuring that the coin 
cannot be spent again. The principle behind this consensus algorithm is that
the user burning the cryptocurrency is showing long-term commitment to the
coin by burnin it. This is because they are taking a short-term loss in 
exchange for a long-term gain.\cite{consensusmedium}




\section{Advantages and Disadvantages}
\label{sec:evaluate}

As a result for its complexity, blockchain's potential as a distributed 
form of record-keeping is almost without limit. But as every other 
technology, even the blockchain have its pros and cons. 

\subsection{Advantages}
\label{sec:advantages}

\textbf{Trustless system and small fees}
In most traditional systems, a consumer pays a bank to verify a transaction,   
a notary to sign a documents, or a minister to perform a marriage. Using 
blockchain technologies, this is no longer necessary because the distributed
nodes verify the transactions through consensus algorithms.\\
Therefore, blockchain systems negate the risk of trusting a single person or 
organization and reduces the overall costs and transaction fees by cutting 
third-parties. For this reason, blockchain is often referred to as a "trustless"
system.\cite{binancevision}\cite{investopedia}
\bigskip\\
\textbf{Improved accuracy by removing human involvement in verification} 
\bigskip\\
\textbf{Transactions are secure, private and efficient}
Also in this case, by eliminating the human factor, we obtain a 24 hours a day 
working system, instead of institutions operating only during business hours.
As a result, transactions can be completed in about ten minutes and can be 
considered secure after just a few hours. Actually, once a transaction is 
recorded, its authenticity must be verified by the blockchain network. Where, 
millions of nodes rush to confirm that the details of the transaction are correct.

Moreover, every block of the chain has the hash of the block before it along with
its own hash, wich is computed with the data contained in the block. So, when
an information in a block is edited in any way, its own hash will change, but the
hash in the block after it would not. This contrast makes extremely hard to 
change data inside the blockchain without beeing noticed.

Considering that the majority of blockchains are pubblic, it means that anyone 
can access the blockchain and read the data inside it with all the transaction 
history. So, if anyone can read see all the transactions, why is people say that
blockchain transactions are anonymous?\\
That’s why anonymity is confused with confidentiality. Indeed, every user makes 
public transactions with their unique code called "public key"\footnote{Public-key cryptography, is a cryptographic system that uses pairs of keys: public keys which may be disseminated widely, and private keys which are known only to the owner.},
which is recorded on the blockchain, rather than their personal information.\cite{investopedia}
\bigskip\\
\textbf{Transparency}
A big benefit to this technology is given by his open source nature. That means
that the users of the blockchain network can modify the code as they see fit, so long
as they have the approval from the majority of the network's nodes.
\bigskip\\
\textbf{Distributed}
Blockchain does not store any of its information in a unique node or central 
location. Instead, data is stored in thousands of devices on a distributed
network. Whenever a block is added to the chain, every node updates its blockchain
to reflect the change. Because of this, there is no single point of failure
and so becomes more difficult to tamper with.\cite{binancevision}\cite{investopedia}
\bigskip\\
\textbf{Stability}
Once a block is confirmed, its extremely difficult to remove or change the data inside it.
This makes blockchain a great technology for storing financial records or any other data 
where an audit trail is required because every change is tracked and permanently recorded 
on a distributed and public ledger.\cite{binancevision}
\bigskip

\subsection{Disadvantages}
\label{sec:disadvantages}

While there are significant upsides to the blockchain, there are also serious 
challenges to its adoption. The most ones are political and regulatory, but 
there are also some technical challages to take in account.
\bigskip\\
\textbf{Energy inefficiency}
Some blockchains, especially those using Proof of Work consensus algorthms \ref{sec:pow}, are 
higly inefficient. Since mining is higly competitive and there is 
only one winner every ten minutes, the work of every other miner is wasted.
As miners are continually trying to increase their computational power, 
so they have a greater chance of finding a valid block hash, the resources used 
by the Bitcoin network has increased significantly in the last few years, and it 
currently consumes more energy than many countries, such as Denmark, Ireland, 
and Nigeria.\cite{binancevision}
\bigskip\\
\textbf{Scalability}
The Bitcoin is a perfect case study for the low scalability of blockchains.
In fact, to add a new block on the blockchain, it takes about ten minutes and
the maximum size of each block is of 2 MegaBytes. This entails that there is a
limited number of transaction per block. It is estimated that the Bitcoin's 
blockchain can manage only seven transactions per second (TPS). Meanwhile, the
Visa's actual infrastructure can process 24.000 TPS.
\bigskip\\
\textbf{51\% Aattacks}
The consensus algorithms that protetc the blockchain has proven to be very 
efficient over the years (thanks to ). However, there are a few potential attacks that can 
be performed against blockchain networks. The most discussed is the  51\% attack.
Such an attack may happen if one entity manages to control more than 50\% of the 
network, which would allow them to edit or modify the order of the transactions.
Altought being theoretically possible, there was never a succesful 51\% attack on 
the Bitcoin blockchain, additionaly, as the network gets larger, the security 
increases.\cite{binancevision}
\bigskip\\
\textbf{Private keys} 
Blockchain uses public-key cryptography to give users ownership over their blockchain data.
Each public-key correspons to a private-key which should be kept secret. Users need their 
private-key to access their data. If a user loses their private key, the money is 
effectively lost, and there is nothing they can do about it.\cite{binancevision}
\bigskip\\
\textbf{Storage}
Distributed ledgers can increase very much over time. For example the Bitcoin blockchain
currently requires 239\footnote{Last update: 10/9/2019} GigaByte of storage. Even if in 
the Bitcoin paper its said that thanks to "Moore's Law"\cite{bitcoin} the storage should
not be a problem, the current growth in blockchain size appears to be outstripping the growth
in hard drives and the network risks losing nodes if the ledger becomes too large for 
individuals to download and store.\cite{binancevision}
\section{Use Cases}
\label{sec:usecases}

The merit to the great attention given to the blockchain technology is to be attributed
to Bitcoin. Indeed, thanks to this digital currencies (cryptocurrencies) the 
blockchain-based solutions have grown. So learning how this innovative technology
can be applied to different scenarious is very important. 

\subsection{Digital Identity}
\label{sec:identity}

The identity component in a blockchain is fulfilled through the use of cryptographic keys.
Combining a public and private key creates a strong digital identity reference based 
on possession of informations or certifications. The public key, is distributed 
publicly and it is what identifies the identity, the private key is kept secret by
the identity and it is used to express consent to digital interactions and to 
sign the data to record in the block.\cite{coindesk}

\subsection{Charity}
\label{sec:charity}

The number of charitable organizations which addopted cryptocurrencies as a donation
method is increasing more and more. The reason is because crypto's transparency,
in fact, thanks to this characteristic, every donation can be tracked and verified.
So, the higher level of transparency and public accountability can ease donors' minds 
and encourage them to give while also reinforcing the charity's reputation for integrity.
Moreover, blockchain technology reduces the fees and taxes to manage every donation by
reducing the number of required intermediaries.\cite{binancevision}

\subsection{Healthcare}
\label{sec:healthcare}

Blockchain may offer significant benefits to hospitals in terms of security, 
interoperability and transparency. Unlike traditional databases that rely on a centralized
server, the use of a distributed system allows for data exchange with higher levels of 
security. In addition, distributed ledgers increase the interoperability among clinics,
hospitals, and other health service providers by allowing all theese parties to interface
with a unique storage system.

Furthermore, blockchain systems may also give higher levels of accesibility and 
transparency over their own health information. But also, allows pharmaceutical 
organizations to increase the efficiency of theri infrastructure by cutting 
down on the widespread problem of drug counterfeiting.\cite{binancevision}

\subsection{Governance}
\label{sec:governance}

The governance can be greatly improven in various sectors by the blockchain technologies.
A short example can be the efficiency of taxes, ideed, because governments rely on 
taxpayer funds, it's important that they use their budgets wisely. Blockchain systems 
and smart contracts can be employed to automate tasks and workflows, which would 
greatly reduce time and money spent on bureaucratic processes.\cite{binancevision}

Another application can be the election process. With the support of a workign digital 
identity system, and the high level of immutability given by the blockchain, this 
technology could be an excellent solution for ensuring that votes cannot be tampered with.
Creating the possibility to transform the secure online voting into a reality. 

\subsection{Payments}
\label{sec:payments}

Thanks to Bitcoin, this is the most widespread use case of the blockchain. It starts in 
2008 when Satoshi Nakamoto published the Bitcoin's white paper, where, for the first time, 
all the above components of a blockchain were used to solve the double-spendign problem.\pagebreak


