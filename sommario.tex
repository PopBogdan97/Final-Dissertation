\chapter*{Abstract} % senza numerazione
\label{abstract}

\addcontentsline{toc}{chapter}{Abstract} % da aggiungere comunque all'indice

The thesis attempts to explain what a blockchain is, 
how it works and why it seems to have so much potential. 
In particular, this debate focuses on the different purposes of this
technology with its functionalities and its implementations by 
analysing two specific blockchains: Bitcoin (BTC)\cite{bitcoin.org} and Bitcoin Cash (BCH)\cite{bitcoincash}. 
The thesis also exposes which are the requirements 
a blockchain should have when it is used as a coin, in order to be 
globally handled as the leading currency. Hence, it goes deeper into the
differences between the BTC and BCH analysing the distinct approaches to 
satisfy the demands of a cryptocurrency. Especially, it illustrates
the main reasons why the Bitcoin hard fork was reached on August 1st, 
2017, giving life to the Bitcoin Cash blockchain.
Once the behaviour of this new technology is clear, the thesis goes 
further with the integration of BCH in the programming language \textit{R} (see section \ref{sec:environment} for more details).
In this way, it provides a means to work with the data stored into the Bitcoin
Cash blockchain and retrieve interesting statistics regarding it.

  In other words, the steps to achieve the goal for this thesis will be:



\begin{itemize}
  \item Understanding and analysing blockchain technologies.
  \item Studying the two most important Bitcoin blockchains: BTC - Bitcoin and BCH - Bitcoin Cash.
  \item Designing a new Package in R for data reading and manipulation of the Bitcoin Cash blockchain.
  \item Implementation of the Package in R with a description of all the possible commands developed.
  \item Example of the Package functionalities and its potential.
\end{itemize}




