\chapter*{Abstract} % senza numerazione
\label{abstract}

\addcontentsline{toc}{chapter}{Abstract} % da aggiungere comunque all'indice

The thesis tries to explain what a blockchain is, 
how it works and why it seems to have so much potential. 
In particular, this debate focuses on the different purposes of this
technology with its functionalities and its implementations by 
analyzing two specific blockchains: Bitcoin (BTC) and Bitcoin Cash (BCH). 
The thesis also exposes which are the requirements 
a blockchain should have when it is used as a coin, in order to be 
globally handled as the leading currency. Hence, it goes deeper into the
differences between the BTC and BCH analyzing the distinct approaches to 
satisfy the demands of a cryptocurrency. Especially, It illustrates
the main reasons why the Bitcoin hard fork was reached on November 15, 
2018, giving life to the Bitcoin Cash blockchain.
Once the behavior of this new technology is clear, the thesis goes 
further with the integration of BCH in the programming language "R".
In this way, it provides a mean to work with the data stored into the Bitcoin
Cash blockchain and retrieve interesting statistics out of it.

  In other words, the steps to achieve the goal for this thesis will be:



\begin{itemize}
  \item Understanding and analyzing blockchain technologies
  \item Studying the two most important Bitcoin blockchains BTC - Bitcoin and BCH - Bitcoin Cash
  \item Design a new Package in R for data reading and manipulation of the Bitcoin Cash blockchain  
  \item Implementation of the Package in R with a description of all the possible commands developed 
  \item Sample of the Package functionalities and its potential
\end{itemize}




